\documentclass[conference]{IEEEtran2/IEEEtran}

%opening
\title{}
\author{}

\begin{document}

\title{Integration of Multiple String Solvers}

\author{
    \IEEEauthorblockN{Siwakorn Srisakaokul, Wing Lam}
    \IEEEauthorblockA{Department of Computer Science, University of Illinois at Urbana-Champaign, Urbana, IL
        \\\{srisaka2, winglam2\}@illinois.edu}
}


\maketitle

\begin{abstract}
blah
\end{abstract}

\section{Introduction}
HiHi

\section{Background}
% Talk about the three solvers.
% Strengths and weaknesses of the solvers
% Statistics for number of tests for each solver

\section{Approach}
% Internal representation that we generate with our inputs and how we convert the
% representation to the three outputs. Provide an cool example if possible.
% Provide the list of commands we support if necessary/helpful.
% Translation for how we got our input (?)

\section{Evaluation}
% Translation for how we got our input (?)
% Show the two charts.
% Show how some of DPRLE's tests do not work for DPRLE
% Timing for how some of Z3's tests takes almost 4 times longer
% to run on Hampi.

\section{Related Work}
% Need to talk about SMT-lib, search on Google scholar for other similar projects to ours

\section{Conclusion}
% Should talk about future work somewhere
% !) Complete translation of all of the solver's tests
% 2) Automatic translation the other solver's language to our language
% 3) Accepting SMT-Lib's format as input to support ALL solvers that support SMT-Lib
% 4) Adding other solvers such as Kaluza
% 5) Constraint Analyzer that uses heuristics to determine which solvers to only run
% 6) Use a constraint generator and test which solver is best at solving such constraints

\bibliographystyle{IEEEtran}
\bibliography{references}

\end{document}
