\section{background}
\label{sec:background}

\newcommand{\dprletests}{16\xspace}
\newcommand{\hampitests}{379\xspace}
\newcommand{\ztests}{245\xspace}

This section describes the constraint solvers \imss supports and the concepts we used to develop \imss. The details of these solvers and and our tools are released on our project website~\cite{imss}.

\subsection{\hampi, \dprle, and \zstr}

\begin{figure}[H]
    \centering
    \begin{tabular}{|l|r|r|r|l|}
        %\toprule
        \hline
        \textbf{Constraint solver} & \textbf{LOC} & \textbf{Tests} & \textbf{Version} \\
        \hline
        \dprle & 2992 & \dprletests & 0.5.0\\
        \hampi & 14197 & \hampitests & 20120213\\
        \zstr & 8616 & \ztests & 1.0.0\\
        \hline
    \end{tabular}
    \caption{Constraint solvers supported by \imss.
        Column ``LOC'' represents the number of lines in the constraint solver's source code.
        Column ``Tests'' represents the number of tests provided by the developers of
        the constraint solvers.
    }
    \label{tab:solvers}
\end{figure}

\imss combines three different constraint solvers, \hampi, \dprle, and \zstr in order to
accept a wider range of languages as input and solve a greater variety of constraints.

\dprle is a decision procedure that solves constraints over regular languages. Regular languages
in \dprle are represented in the form of finite state automatas. \dprle constraints can express
membership and concatenation of regular languages~\cite{dprle2009}.
\imss currently supports the latest stable release of \dprle. This version of \dprle
includes \dprletests tests and all \dprletests tests were evaluated by \imss for our evaluation.

\hampi is a string constraint solver over fixed-size string variables. \hampi constraints can express
membership and may contain context-free language definition, strings, and regular-language
definitions~\cite{hampi2009}. Even though \hampi is a string solver over fixed-size string variables,
users can specify a range of lengths to \hampi, and \hampi will try all the lengths in the given range
and output the shortest string that satisfies the constraints.
\imss currently supports the latest stable release of \hampi. This version of \hampi
includes \hampitests tests and a subset of the \hampitests tests were evaluated by \imss for our evaluation.

\zstr, an extension of the Z3 SMT solver, is a string constraint solver that treats strings
as a primitive type and also supports booleans, integers, and string constants.
\zstr constraints can express variety of functionality such as concatenation, substring,
indexof, and replace~\cite{z32013}.
\imss currently supports the latest stable release of \zstr that was built with version
4.1.1 of Z3 (the most up to date version of Z3 \zstr is recommended to be built with).
This version of \zstr
includes \ztests tests and a subset of the \ztests tests were evaluated by \imss for our evaluation.

Given a set of constraints to \imss, \imss will translate the constraints to the format required by these three string solvers. These three string solvers will then output a string that satisfies the given constraints, or report that it is unsatisfiable.

\subsection{Abstract Syntax Tree}
Compilers typically consist of five main steps: lexical analysis, syntactic analysis, semantic
analysis, code generation, and optimization. Lexical analysis parses the source code which converting
a sequence of characters into a sequence of tokens. Then syntactic analysis will build a
Abstract Syntax Tree (AST) from the sequence of tokens. The AST is used to represent the structure
of the program. Each node of the tree denotes a statement or an expression in the source code.
The AST of the source code will be used in the next step, semantic analysis.
Semantic analysis, code generation, and optimization will do semantic checking, such as type checking,
then generate machine code instructions for the AST. Our tool, \imss, performs the first two steps.
The tool does not do type checking or code generation. Instead \imss takes a input constraint which can be
viewed as source code and translates it to a different format for the same source code.
It does this through lexical and syntactic analysis to generate an AST and then translating the AST (which captures all information of the source code) into three input formats
for the three solvers.
