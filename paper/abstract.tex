\begin{abstract}
Many constraint solvers exist, each with their own strengths and weaknesses.
The burden to determine which solver is better equipped to handle regular expressions,
concatenations, context free grammar, and more has been on the users of these constraint
solvers. Not only are users expected to determine which solver may be compatible with
the constraints they would like to solve, users must learn how to express the
constraints they would like to solve in the language that corresponds to the constraint
solvers they would like to use.

A library has been developed to assist users from having to learn the language that
corresponds to the constraint solvers they would like to use.
This library proposes a common input and output for all constraint solvers.
However, this library requires the
developers of constraint solvers to integrate the library with their solvers.
The requirement of this cumbersome integration by developers results in
the limited adoption of this library. Not to mention the fact that even if a
constraint solver integrates such a library, it still continues to burden users to
determine which solver is best at solving particular constraints.

We designed and implemented \imss (Integration of Multiple String Solvers), a tool that
leverages the functionality of existing
constraint solvers to solve any constraint that existing solvers can solve in the
shortest amount of time. \imss accepts a set of constraints as input and translates the
input constraints into the language for other constraint solvers. Once the input
constraints are translated, \imss runs the constraint solvers it supports concurrently
to return satisfied or unsatisfied results to the user in the shortest amount of time.

\imss is lightweight and efficient, and can be easily enhanced to support
any constraint solver.
Our experiments use \imss with 3 renowned constraint solvers and compare each of
the constraint solvers' performance and results with other solvers.
\imss's source code, documentation, and the
experimental data are available on our project website~\cite{imss}.

%The language for \imss's input is expressive and efficient and can successfully
%represent any other constraint solver's language.

\end{abstract}
