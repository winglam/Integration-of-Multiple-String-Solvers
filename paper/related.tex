\section{related works}
\label{sec:related}

Boolector~\cite{boolector2009} is an efficient SMT solver for the quantifier-free theory
of bit-vectors in combination with the extensional theory of arrays.
Boolector depends on term rewriting and bit-blasting for bit-vectors. It
takes a formula expressed in the SMT-LIB format~\cite{smtlib2015}, or alternatively in
the BTOR format~\cite{btor2008}, as input.
The BTOR format is a low-level bit-vector format with clean semantics that is easy to
parse. Additionally, BTOR supports bit-vector arrays and model checking of safety
properties.
Aside from being an SMT solver, Boolector contains a tool known as
Pretty Printer. Pretty Printer allows Boolector to convert formulas from BTOR to
SMT-LIB format and vice versa. The Pretty Printer can be combined with Boolector’s
rewriting module to internally simplify the formula before conversion.
Besides \imss, Pretty Printer is the only existing tool that we know of capable of
converting from one language format to another for constraint solving purposes.

A library has been developed to assist users from having to learn the language that
corresponds to the constraint solvers they would like to use.
This library proposes a common input and output for all constraint solvers.
However, this library requires the
developers of constraint solvers to integrate the library with their solvers.
The requirement of this cumbersome integration by developers results in
the limited adoption of this library. Not to mention the fact that even if a
constraint solver integrates such a library, it still continues to burden users to
determine which solver is best at solving particular constraints.