\section{implementation}
\label{sec:implementation}
In this section we will explain in-depth the implementation of \imss.
The architecture of \imss is shown
in Figure~\ref{fig:imss}. The tool takes string constraints in \imss format
and generates at least one output that satisfies the given constraints if the constraints are satisfiable.
\imss is implemented in Java and uses three different solvers to satisfy its input constraints.


\subsection{\imss's Grammar}
Since \hampi, \dprle, and \zstr takes their inputs in
 different languages, these three languages do not have the same expressiveness.
 We cannot translate everything from one language into the other two languages. For example,
 \dprle, requires automata descriptions for its input~\cite{lazystrings2010}.
 To define a string, we have to construct a finite automaton representing that string.
 Also, \dprle, does not support constraints which uses context-free grammar,
 but \hampi does. However, \dprle, can check whether one regular expression language
 contains another regular expression language. Each language has some functionalities the
 other solvers do not have. Thus, we construct a new grammar (Grammar~\ref{fig:grammar}) that is expressive enough to
 cover all the functionalities in all three languages by combining their grammars.

\subsection{Input Format Translation}
After \imss takes an input constraint in \imss's format, it does lexical analysis to parse
the text-based input into tokens and constructs an AST representation of the input.
Each AST node stores its type, according to the grammar, and pointers to its child nodes.
For example, the AST of \CodeIn{LessThan(a, 2)} consists of three nodes. The root node is of type
\CodeIn{LessThan} and has two children. In this example, the first child is of type \CodeIn{ID} (variable name), and
the second child is of type \CodeIn{INT_LIT} (integer literal). We infer the types from \imss's grammar.
With this approach, every tree node has an associated type. Given the type information of
each AST node, we can easily translate this tree into the three different languages by traversing
on the tree and recursively constructing the input constraints for each of the three languages.
For some cases, there is no direct mapping from \imss language to the language of
a solver.
For example, \dprle does not have a \CodeIn{StartsWith} function. Luckily \imss is capable of coming up with an equivalent
expression in \dprle.

\CodeIn{StartsWith($a$, "abc")} $\iff$ $a <$ Automaton of Regex $abc(.)^{*}$

\imss's output for the \CodeIn{StartsWith} example above says that $a$ is a member of a regular language that starts with $abc$, the subsequent expression symbols $(.)^{*}$
indicates that the satisfying string can have any other characters after the characters
$abc$.

If it is not possible to map a particular function from \imss to the language(s) of the solvers that we support, \imss will not generate an input
for that solver. In other words, the input constraint is incompatible with that solver.

\subsection{Result Aggregation}
After \imss generates the input constraints for each language, it executes the three solvers concurrently, and
reports a string that satisfies the given constraint, assuming at least one solver can handle the input.
