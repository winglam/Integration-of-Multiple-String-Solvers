\section{implementation}
\label{sec:implementation}
We introduce \imss (Integration of Multiple String Solvers) tool as
an implementation of our approach. The architecture of \imss is shown
in Figure~\ref{fig:imss}. The tool takes string constraints in \imss format
and generates a string that satisfies the given constraints. \imss is
implemented in Java and uses the three different solvers to the constraints.
We briefly describe the details in the following three different sections.

\subsection{\imss's Grammar}
Since, \hampi, \dprle, and \zstr take inputs in
 different languages. These three languages do not have the equivalent expressiveness.
 We cannot use one language and translate it into the other two languages. For example,
 \dprle, requires automata descriptions for its input~\ref{lazystrings2010}.
 To define a string, we have to construct a finite automaton representing that string.
 Also, \dprle, does not support a constraint which has a context-free grammar,
 but \hampi does. However, \dprle, can check whether one regular expression language
 contains another regular expression language. Each language has some functionalities which
 other solvers do not have. Thus, we construct a new grammar (Grammar~\ref{fig:grammar}) that is expressive enough to
 cover all the functionalities in all three languages by combining their grammars.

\subsection{Input Format Translation}
After \imss takes a input constraint in \imss's format, it constructs a AST representing the input.
Each AST node stores its type according to the grammar and pointers to its child nodes.
For example, the AST of \CodeIn{LessThan(a, 2)} consists of three nodes. The root node is of type
\CodeIn{LessThan} and has two children. The first child is of type \CodeIn{ID} (variable name), and
the second child is of type \CodeIn{INT_LIT}. We infer the types from the \imss's grammar.
With this method, every tree node has an associated type. Given the type information of
each AST node, we can easily translate this tree into the three different languages by traversing
on the tree and recursively construct input constraints for each languages.

\subsection{Result Aggregation}
After \imss generates the input constraints for each language, it executes the three solvers concurrently, and
reports a string that satisfies the given constraint, if at least one solver can handle the input.
