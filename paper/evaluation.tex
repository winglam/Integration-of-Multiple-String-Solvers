\section{evaluation}
\newcommand{\totaltests}{640\xspace}
\newcommand{\pickedtests}{131\xspace}
\newcommand{\dprlePtests}{16\xspace}
\newcommand{\hampiPtests}{69\xspace}
\newcommand{\zPtests}{46\xspace}

This section describes the setup we evaluated \imss in and the results of our evaluation procedure. More details about the tests we selected for evaluation and their results are made available on our project website~\cite{imss}.

\label{sec:evaluation}
\subsection{Evaluation Setup}
We evaluated our tool with the constraints derived from the tests of the three solvers.
There are \totaltests tests in total (Figure~\ref{tab:solvers}). Of the \totaltests
tests, we evaluated \pickedtests tests. The \pickedtests tests are composed of
\dprlePtests tests from \dprle, \hampiPtests tests from \hampi, and \zPtests from \zstr.
We evaluated only a subset of tests from \hampi and \zstr because the remaining tests
from these two solvers test the same functionality but with different values.
Since the goal of \imss is to ensure that all functionality that is available
from the three constraint solvers is made available through \imss as well, we
omitted tests that did not test additional functionality than the ones we have
selected.

In selecting tests from \zstr and \hampi,
we use the following approach:
\begin{enumerate}
    \item Go through all the tests and categorize them based on their testing functionality.
    \item Count the number of tests in each category.
    \item Reduce tests:
        \begin{itemize}
            \item If the tests contains a functionality or a combination of functionalities that have not been covered, select those tests for evaluation.
            \item Otherwise, omit them from the list of tests for evaluation.
        \end{itemize}
\end{enumerate}
After we have the set of tests, we manually translate each of the tests to be in the
format supported by \imss and run \imss on each of them to collect the following results.
The machine we used to run \imss has the following specifications: 2.8 GHz Intel Core i7, 4.8GB of RAM, and 4 processors in version 14.04 LTS of Ubuntu Desktop.

\subsection{Results}
\begin{figure}[H]
    \centering
    \begin{tabular}{|l|c|c|c|}
        \hline
        \textbf{Constraint solver} & \dprle & \hampi & \zstr \\
        \hline
        \multicolumn{4}{|l|}{}  \\
        \multicolumn{4}{|l|}{\textbf{\dprle tests}}  \\
        \hline
        SAT & 14 & 2 & 2 \\
        UNSAT & 0 & 0 & 0 \\
        Incompatible & 2 & 14 & 14 \\
        \hline
        \multicolumn{4}{|l|}{}  \\
        \multicolumn{4}{|l|}{\textbf{\hampi tests}}  \\
        \hline
        SAT & 3 & 49 & 12 \\
        UNSAT & 0 & 20 & 2 \\
        Incompatible & 66 & 0 & 55 \\
        \hline
        \multicolumn{4}{|l|}{}  \\
        \multicolumn{4}{|l|}{\textbf{\zstr tests}}  \\
        \hline
        SAT & 5 & 3 & 38 \\
        UNSAT & 0 & 1 & 8 \\
        Incompatible & 41 & 42 & 0 \\
        \hline
    \end{tabular}
    \caption{
        For each solver's set of tests, this table shows the number of tests each solver
        is incompatible with and for compatible tests, what the output of those tests
        are.
    }
    \label{tab:solvercompareresults}
\end{figure}

\begin{figure}[H]
    \centering
    \begin{tabular}{|l|c|c|c|}
        \hline
        \textbf{Constraint solver} & \dprle & \hampi & \zstr \\
        \hline
        \dprle & 1.00 & 30.52 & 13.03 \\
        \hampi & 0.04 & 1.00 & 0.40 \\
        \zstr & 0.31 & 11.91 & 1.00 \\
        \hline
    \end{tabular}
    \caption{
        This table shows the increase or decrease in time for the solvers to solve the
        same constraints.
        The data in each cell is calculated by taking the average time the solver
        represented by the row took to solve the tests belonging to the solver represented
        by the column over the average time the solver represented by the column
        took to solve the same tests. A number less than 1 represents a decrease in time.
    }
    \label{tab:solvercomparetime}
\end{figure}
