\section{evaluation}
\label{sec:evaluation}
\subsection{Evaluation Setup}
We test our tool with the constraints derived from the tests from the three solvers.
There are 640 tests in total (Figure~\ref{tab:solvers}). We picked 131 tests:
all 16 tests from \dprle, 46 from \zstr, and 69 from \hampi. In selecting tests from \zstr and \hampi,
we use the following approach:
\begin{enumerate}
    \item Go through all the tests and categorize them based on their testing functionality.
    \item Count the number of tests in each category.
    \item Reduce tests:
        \begin{itemize}
            \item If the number of tests is less or equal to 10, pick the first, middle, and last tests.
            \item If the number of tests is less or equal to 50, pick only 20\% of them.
            \ping{TODO: Wing, I don't know how to explain: picking 1, 20, 30, ... here}
            \item Otherwise, pick only 10\% of them.
        \end{itemize}
\end{enumerate}
After we have the set of tests, we run \imss on each of them and collect the results.
\begin{figure}[H]
    \centering
    \begin{tabular}{|l|c|c|c|}
        \hline
        \textbf{Constraint solver} & \dprle & \hampi & \zstr \\
        \hline
        \multicolumn{4}{|l|}{}  \\
        \multicolumn{4}{|l|}{\textbf{\dprle tests}}  \\
        \hline
        SAT & 14 & 2 & 2 \\
        UNSAT & 0 & 0 & 0 \\
        Incompatible & 2 & 14 & 14 \\
        \hline
        \multicolumn{4}{|l|}{}  \\
        \multicolumn{4}{|l|}{\textbf{\hampi tests}}  \\
        \hline
        SAT & 3 & 49 & 12 \\
        UNSAT & 0 & 20 & 2 \\
        Incompatible & 66 & 0 & 55 \\
        \hline
        \multicolumn{4}{|l|}{}  \\
        \multicolumn{4}{|l|}{\textbf{\zstr tests}}  \\
        \hline
        SAT & 5 & 3 & 38 \\
        UNSAT & 0 & 1 & 8 \\
        Incompatible & 41 & 42 & 0 \\
        \hline
    \end{tabular}
    \caption{
        For each solver's set of tests, this table shows the number of tests each solver
        is incompatible with and for compatible tests, what the output of those tests
        are.
    }
    \label{tab:solvercompareresults}
\end{figure}

\begin{figure}[H]
    \centering
    \begin{tabular}{|l|c|c|c|}
        \hline
        \textbf{Constraint solver} & \dprle & \hampi & \zstr \\
        \hline
        \dprle & 1.00 & 30.52 & 13.03 \\
        \hampi & 0.04 & 1.00 & 0.40 \\
        \zstr & 0.31 & 11.91 & 1.00 \\
        \hline
    \end{tabular}
    \caption{
        This table shows the increase or decrease in time for the solvers to solve the
        same constraints.
        The data in each cell is calculated by taking the average time the solver
        represented by the row took to solve the tests belonging to the solver represented
        by the column over the average time the solver represented by the column
        took to solve the same tests. A number less than 1 represents a decrease in time.
    }
    \label{tab:solvercomparetime}
\end{figure}
